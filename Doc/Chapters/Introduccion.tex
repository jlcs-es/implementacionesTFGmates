%************************************************
\chapter{Introducción}\label{ch:introduction}
%************************************************

Estructura:
\begin{itemize}
	\item Preliminares: Teoría ya dada en la carrera y que no necesita demostraciones
		\subitem Preliminares de computación: introducir problemas de decisión, clases de complejidad (para justificar qué problemas se usan en criptografía).
		\subitem Preliminares de álgebra: grupos y anillos, congruencias, id. de Bezout y alg. de Euclides extendido para los inversos. El problema del logaritmo discreto.
		\subitem Preliminares de grafos: definiciones básicas, coloración de grafos, problemas del isomorfismo de grafos y la 3-coloración.
		
	\item Residuos cuadráticos: indicar que es teoría de álgebra que no se da en la carrera y por eso dedicamos un capítulo a sus resultados.
		\subitem Definición, propiedades, símbolo de Legendre, símbolo de Jacobi, alg. raíces cuadradas, problema de residuosidad cuadrática.
		
	\item ZKP
		\subitem La historia de la cueva.
		\subitem Pruebas interactivas: completitud y robustez (soundness).
		\subitem ZKP: simulador.
			\subsubitem ZKP con residuos cuadráticos
			\subsubitem ZKP con isomorfismo de grafos
			\subsubitem ZKP con logaritmo discreto
		\subitem Otros tipos de ZKP: estadísticos, computacionales.
		
	\item Aplicaciones ZKP
		\subitem Firma utilizando un hash en vez de reto
		\subitem Protocolos de identificación
			\subsubitem Fiat-Shamir
			\subsubitem Feige-Fiat-Shamir
			\subsubitem Schnorr
			
			
	\item Implementaciones
\end{itemize}