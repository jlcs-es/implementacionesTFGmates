%************************************************
\chapter{Residuos Cuadráticos}\label{ch:qr} 
%************************************************

%TODO
TODO : Párrafo de introducción al capítulo y referencias
% Indicar que la teoría de QR no es vista y sí se desarrolla

% ref
% Handbook of applied
% book:544293
% A course in computational
% A Classical Introduction to Modern Number Theory
% Proofwifi ?
% http://www2.math.ou.edu/~kmartin/nti/

Teoría de símbolos de Lebesgue, ..., residuos cuadráticos, cálculo de raíz discreta?


%%%%%%

%TODO: número de raíces cuadradas de un número, y demostrar que nºraíces_cuadradas_de_un_numero X nºresiduos_cuadraticos = |Z_n^* |

% TODO: propiedad de los primos pq congruentes con 3 mod 4 que hace que el -1 es un no-residuo cuadrático mod n=pq, con símbolo de jacobi +1. n es un entero de Blum.

\section{Primeras propiedades}

\begin{definition}
	Sea $a\in \mathbb{Z}^*_n$. Se dice que $a$ es un \textit{residuo cuadrático} módulo n, o un \textit{cuadrado} módulo n, si existe un $x \in \mathbb{Z}^*_n$ tal que $x^2 \equiv a \, mod \, n$.
	Si no existe dicho $x$, entonces $a$ se llama un \textit{no-residuo cuadrático} módulo n.
	
	Al conjunto de todos los residuos cuadráticos módulo n de $\mathbb{Z}^*_n$ los denotaremos como $Q_n$ o bien como $\mathbb{Z}^{Q+}_n$.
	Al conjunto de los no-residuos cuadráticos lo denotamos como $\overline{Q_n}$.
\end{definition}

\begin{example}
	Si tomamos $n=4$, los no-residuos cuadráticos son $2$ y $3$, y el único residuo cuadrático es $1$:
	\begin{align*}
		1^2 \equiv 1 \, mod \, 4 \qquad 2^2 \equiv 0 \, mod \, 4 \qquad  3^2 \equiv 1 \, mod \, 4 
	\end{align*}
	
\end{example}


\begin{remark}
	Por definición $0 \notin \mathbb{Z}^*_n$, y por tanto $0 \notin Q_n$ ni $0 \notin \overline{Q_n}$.
\end{remark}



\begin{definition}
	Sea $a \in Q_n$. Si $x \in \mathbb{Z}^*_n$ satisface $ x^2 \equiv a \, mod \, n$, entonces $x$ se llama \textit{raíz cuadrada} módulo n de $a$.
\end{definition}



\begin{proposition}
	Sea $p$ un primo impar. Se cumple que $|Q_p| = \frac{p-1}{2}$ y $|\overline{Q_p}| = \frac{p-1}{2}$, es decir, la mitad de los elementos de $\mathbb{Z}^*_p$ son residuos cuadráticos, y la otra mitad no-residuos cuadráticos.
\end{proposition}

\begin{proof}
	Sea $\alpha \in \mathbb{Z}^*_p$ un generador de $\mathbb{Z}^*_p$.
	Un elemento $a \in \mathbb{Z}^*_p$ es un residuo cuadrático módulo p si y solo si $a \equiv \alpha^i \, mod \, p$ donde $i$ es un entero par. Como $p$ es primo, $\phi(p) = p-1 = |\mathbb{Z}^*_p|$, que es un entero par, y de ahí se sigue el enunciado, la mitad de los elementos son generados por un $i$ par, la otra mitad por un $i$ impar.
\end{proof}

\begin{example}
	Para $p=13$ tenemos que $\alpha = 6$ es un generador de $\mathbb{Z}^*_{13}$. Las potencias de $\alpha$ módulo 13 son:
	
	\begin{tabular}{|c||c|c|c|c|c|c|c|c|c|c|c|c|}
		 \hline
			$i$ & $1$ & $\underline{2}$ & $3$ & $\underline{4}$ & $5$ & $\underline{6}$ & $7$ & $\underline{8}$ & $9$ & $\underline{10}$ & $11$ & $\underline{12}$ \\
			\hline
			$6^i \, mod \, 13$ & $6$ & $10$ & $8$ & $9$ & $2$ & $12$ & $7$ & $3$ & $5$ & $4$ & $11$ & $1$ \\
		 \hline
	\end{tabular}

	\hfil

	Lo que nos da $Q_{13} = \{1,\,3,\,4,\,9,\,10,\,12\}$ y $\overline{Q_{13}} = \{2,\,5,\,6,\,7,\,8,\,11\}$.
\end{example}

\begin{proposition}
	\label{numResCuadpq:prop}
	Sea $n$ un producto de dos primos impares $p$ y $q$, $n = pq$. Entonces  $a \in \mathbb{Z}^*_p$ es un residuo cuadrático módulo $n$, $a \in Q_n$ si y solo si $a \in Q_p$ y $a \in Q_q$. Se sigue que $|Q_n| = |Q_p|\cdot |Q_q| = \frac{(p-1)(q-1)}{4}$, y por tanto  $\overline{|Q_n|} = |\mathbb{Z}^*_n| - |Q_n| = \frac{3(p-1)(q-1)}{4}$.
\end{proposition}

\begin{proof}	
	Si $a$ es un residuo cuadrático módulo $n=pq$, $a \equiv x^2  \, mod \, n$, es inmediato que en módulos $p$ y $q$ se cumple $a \equiv x^2  \, mod \, p$, $a \equiv x^2  \, mod \, q$.
	
	Si tenemos que $a$ es un residuo cuadrático módulo $p$, $a \equiv x_p^2 \, mod \, p$, y también módulo $q$, $a \equiv x_q^2 \, mod \, q$, por el Teorema Chino de los Restos existe un $x$ tal que:
	
	$x \equiv x_p \, mod \, p$ \quad y \quad $x \equiv x_q \, mod \, q$
	
	De modo que, elevando al cuadrado:
	
	$x^2 \equiv x_p^2 \equiv a \, mod \, p$
	
	$x^2 \equiv x_q^2 \equiv a \, mod \, q$
	
	Por lo que $x^2 \equiv a \, mod \, n$.
	
\end{proof}



% TODO: ¿hace falta? --> Sí
%\begin{proposition}
%	Si $p$ es un primo impar y $a \in Q_p$, entonces $a$ tiene exáctamente $2$ raíces módulo $p$.
%\end{proposition}





\section{Símbolo de Legendre}

Para identificar los residuos cuadráticos disponemos de una herramienta muy útil:

\begin{definition}
	Dados un primo impar $p$ y un entero $a$, se define el \textit{símbolo de Legendre} $\left( \frac{a}{p} \right) $ como
	
	\begin{center}
		$
		\left( \dfrac{a}{p} \right) = 
		\begin{cases}
			0, & si\ a \equiv 0 \, mod \, p\\
			1, & si\ a \in Q_p  \\
			-1, & si\ a \in \overline{Q_p} \\
		\end{cases}
		$
	\end{center}
\end{definition}

\hfil

Veamos ahora algunas propiedades del símbolo de Legendre:

\begin{theorem}[Criterio de Euler]
	Sea $p$ un primo impar. Sea $a \not\equiv 0 \, mod \, p$. Entonces:
	
 	\begin{center}
 		$ \left( \dfrac{a}{p} \right)  \equiv a^{(p-1)/2}  \, mod \, p$
 	\end{center}
	
\end{theorem}

\hfil

\begin{proof}
	Observemos primero que las raíces de $1$ módulo $p$ son $1$ y $-1$ $mod\, p$. También que por el Teorema de Euler, $a^{\phi(p)} \equiv a^{p-1} \equiv 1 \, mod \, p$.
	
	De este modo, tenemos que $ a^{\frac{p-1}{2}} \equiv \pm 1 \, mod \, p$.
	
	Ahora demostrar el teorema es equivalente a demostrar que $a^{(p-1)/2} \equiv 1 \, mod \, p$ si y solo si $a$ es un residuo cuadrático.
	
	\hfil
	
	Supongamos que $a$ es un residuo cuadrático módulo $p$. Sea $x$ tal que $x^2 \equiv a \, mod \, p$. Entonces, $ a^{\frac{p-1}{2}} \equiv x^{(p-1)} \equiv 1 \, mod \, p$, de nuevo por el Teorema de Euler.
	
	\hfil
	
	Sea ahora  $ a^{\frac{p-1}{2}} \equiv 1 \, mod \, p$.
	
	Tomamos $g$ un generador de $\mathbb{Z}^*_p$, de modo que $a \equiv g^r \, mod \, p$.
	
	Sustituyendo:
	 $ g^{r\frac{p-1}{2}} \equiv 1 \, mod \, p$, 
	 y utilizando que $g$ tiene orden $p-1$, nos queda $ g^{r\frac{p-1}{2}} \equiv g^{\frac{r}{2}} \equiv 1 \, mod \, p$, de donde deducimos que necesariamente $r$ es un entero par, $r = 2s$.
	 
	 Construimos $x \equiv g^s \, mod \, p$, que cumple:
	 $x^2 \equiv g^{2s} \equiv g^r \equiv a \, mod \, p$,
	 de modo que $a$ es un residuo cuadrático módulo $p$.
	
\end{proof}


\begin{example}
	Sea $p=13$ y $a=5$, como $5^{11}\equiv -1 \, mod \, 23$, por el criterio de Euler, $\left( \frac{5}{23} \right) = -1$, por lo que $5$ es un no-residuo cuadrático de $23$.
\end{example}


\begin{proposition}
	Sean $p$ un primo impar, $a,b\in \mathbb{Z}_p$.
	Si $a\equiv b \, mod \, p$, entonces $\left( \frac{a}{p} \right) = \left( \frac{b}{p} \right)$.
\end{proposition}


\begin{proof}
	Si $a$ es residuo cuadrático, entonces existe $x\in \mathbb{Z}_p^*$ tal que $x^2 \equiv a\equiv b \, mod \, p$, y $b$ es también residuo cuadrático. Análogo en el caso contrario.	
\end{proof}


\begin{proposition}[Propiedad multiplicativa del símbolo de Lebesgue]
	Sean $a$ y $b$ enteros coprimos con $p$, un primo impar. Entonces:
	
\begin{center}
 	$
		\left( \dfrac{a}{p} \right) 	\left( \dfrac{b}{p} \right) = 	\left( \dfrac{ab}{p} \right) 
	$.
\end{center}
	
	En particular, el producto de dos no-residuos cuadráticos es un residuo cuadrático.
	
\end{proposition}

\begin{proof}
	Utilizando el Criterio de Euler:
	
	\[
			\left( \dfrac{a}{p} \right) 	\left( \dfrac{b}{p} \right) \equiv
			a^{(p-1)/2} \cdot b^{(p-1)/2} \, mod \, p \equiv (ab)^{(p-1)/2} \, mod \, p \equiv	\left( \dfrac{ab}{p} \right) 
	\]
	
\end{proof}

\begin{corollary} Sea $p$ primo impar, $a \in \mathbb{Z}_p^*$, entonces
	$\left( \dfrac{a^2}{p} \right) = 1$
\end{corollary}

\begin{proof}
	Como $\left( \frac{a}{p} \right) = \pm 1$, $\left( \frac{a^2}{p} \right) = \left( \frac{a}{p} \right)\cdot \left( \frac{a}{p} \right) = 1$.
\end{proof}

% TODO : hace falta? -> sí, para Jacobi
% https://proofwiki.org/wiki/Law_of_Quadratic_Reciprocity
\begin{theorem}[Ley de reciprocidad cuadrática]
	Sean $p$ y $q$ primos impares distintos, se cumple:
	
	\begin{center}
		$
	\left( \dfrac{p}{q} \right) 	\left( \dfrac{q}{p} \right) = \left( -1 \right) ^{(p-1)(q-1)/4}
	$.
	\end{center}
	O de otro modo:
	
	\begin{center}
		$
		\left({\dfrac p q}\right) = \begin{cases}
		\quad \left({\dfrac q p}\right) & Si\quad p \equiv 1 \, mod \, 4 \quad  ó \quad  q \equiv 1 \, mod \, 4 \\
		-\left({\dfrac q p}\right) & Si\quad p \equiv q \equiv 3 \, mod \, 4.
		\end{cases}
		$
	\end{center}
	
	\label{quadRec:theo}
	
	%TODO : las fórmulas extra del -1 y 2
	
\end{theorem}



\begin{proposition}[Primera ley suplementaria]
	Sea $p$ primo impar. Entonces:
	
	\begin{center}
		$
			\left( \dfrac{-1}{p}\right) = (-1)^{(p-1)/2}
		$
	\end{center}
	
	
	O de otro modo:
	
	\begin{center}
		$
		\left(\dfrac{-1}{p}\right) = \begin{cases}
		\quad 1 & Si\quad p \equiv 1 \, mod \, 4 \\
		-1 & Si\quad p \equiv 3 \equiv -1 \, mod \, 4.
		\end{cases}
		$
	\end{center}
	
\end{proposition}

\begin{proof}
	La primera expresión es inmediata por el criterio de Euler:
	\begin{center}
	$
	\left( \dfrac{-1}{p}\right) = (-1)^{(p-1)/2} \, mod \, p
	$
	\end{center}	
	y nos basta hacer un análisis de casos respecto a la congruencia de $p \, mod \, 4$.
	
	Como $p$ es primo impar, no será congruente con $\overline{0}$ ni $\overline{2}$.
	
	Si $p \equiv 1 \, mod \, 4$, podemos escribir $p = 4k+1$ para algún entero $k$. Entonces,
	\begin{center}
	$
	(-1)^{(p-1)/2} = (-1)^{2k}  = 1,
	$
	\end{center}
	por lo que $\left( \frac{-1}{p}\right) = 1$.

	Y si $p \equiv 3 \, mod \, 4$, podemos escribir $p = 4k+3$ para algún entero $k$. Entonces,
	\begin{center}
		$
		(-1)^{(p-1)/2} = (-1)^{2k+1}  = -1,
		$
	\end{center}
	y tenemos que $\left( \frac{-1}{p}\right) = -1$.
\end{proof}



\begin{proposition}[Segunda ley suplementaria]
	Sea $p$ primo impar. Entonces:
	
	\begin{center}
		$
		\left( \dfrac{2}{p}\right) = (-1)^{(p^2-1)/8}
		$
	\end{center}
	
	
	O de otro modo:
	
	\begin{center}
		$
		\left(\dfrac{-1}{p}\right) = \begin{cases}
		\quad 1 & Si\quad p \equiv \pm 1 \, mod \, 8 \\
		-1 & Si\quad p \equiv \pm 3 \, mod \, 8.
		\end{cases}
		$
	\end{center}
	
\end{proposition}
% Pruebas por Lema de Gauss, análisis de casos, etc.





\hfil

\section{Símbolo de Jacobi}

% TODO: la parte de Kronecker no haría falta
% TODO: notación Z^Q para designar símbolo Jacobi = 1, Z^Q = Z^Q+ U Z^Q-

El símbolo de Legendre está definido para módulos un primo impar $p$. Ahora vamos a ver una generalización del concepto:

\begin{definition}
	Sean $a,\,N\in \mathbb{Z}$, con $N = p_1 p_2 \cdots p_r$, donde los $p_i$ son primos impares, no necesariamente distintos.
	
	Definimos el \textit{Símbolo de Kronecker-Jacobi} $\left( \frac{a}{N} \right) $ como
		\begin{center}
			$
			\left( \dfrac{a}{N} \right) = \prod_{i=1}^{r} \left( \dfrac{a}{p_i} \right)
			$
		\end{center}
		
	donde $\left( \frac{a}{p_i} \right)$ es el Símbolo de Legendre.
	 
\end{definition}

\begin{remark}
	A diferencia del Símbolo de Legendre, el Símbolo de Jacobi $\left( \frac{a}{N} \right) $ no indica si $a$ es un residuo cuadrático módulo $N$. Es cierto que si $a \in Q_N$, su Símbolo de Jacobi será $\left( \frac{a}{N} \right) = 1$, pero el contrario no se cumple.
\end{remark}

\begin{example}
	Sea $a=2$ y $N=15=3\cdot 5$.
	\begin{center}
		$\left( \dfrac{2}{15} \right) = \left( \dfrac{2}{3} \right) \left( \dfrac{2}{5} \right) =  \left(-1\right) \cdot  \left(-1\right) = 1$
	\end{center}
	Ya que $Q_3 = \{1\}$, $\overline{Q_3}=\{2\}$, y $Q_5 = \{1,4\}$, $\overline{Q_5}=\{2,3\}$.
\end{example}

\hfil

Igual que antes, veamos algunas propiedades del Símbolo de Jacobi:


\begin{theorem}
	Propiedades del Símbolo de Jacobi:
	
	\begin{enumerate}[label=(\roman*)]
		\item Si $a \equiv b \, mod \, N$, entonces $\left( \frac{a}{N} \right) = \left( \frac{b}{N} \right)$
		
		\item 	$\left( \frac{a}{N} \right) = 0$ si y sólo si $mcd(a, N) \neq 1$.
		\item Para cada $a$, $b$ y $c$ enteros, tenemos: \\
		
		\begin{center}
			$
			\left( \dfrac{ab}{c} \right) = \left( \dfrac{a}{c} \right) \left( \dfrac{b}{c} \right), \quad \left( \dfrac{a}{bc} \right) = \left( \dfrac{a}{b} \right) \left( \dfrac{a}{c} \right) \quad si\ bc \neq 0
			$.
		\end{center}
		
		%\item Fijado $N > 0$, el símbolo $ \left( \frac{a}{N} \right) $ es periódico en $a$ con periodo $N$ si $N \not\equiv 2 \, mod \, 4$, en otro caso, es periódico con periodo $4N$.
		
		%\item Fijado $a \neq 0$, el símbolo $ \left( \frac{a}{N} \right) $ es periódico en $N$ con periodo $|a|$ si $a \equiv 0 \ ó \ 1 \, mod \, 4$, en otro caso, es periódico con periodo $4|a|$.
		
		\item Las fórmulas del \autoref{quadRec:theo} se siguen verificando si $p$ y $q$ son enteros impares positivos, ya no necesitan ser primos.
		
	\end{enumerate}
\end{theorem}

%TODO: proof


\section{El problema de residuosidad cuadrática}

Podemos introducir ahora  el problema de decisión QR, donde dado un módulo $N$, compuesto e impar, decidir si un entero $x$ con símbolo de Jacobi $1$ respecto a $N$, es o no un residuo cuadrático:

\hfil

\begin{tabular}{|ll}
	\textit{Nombre:} & Problema de residuosidad cuadrática (QR). \\
	\textit{Parámetros:} & Un entero compuesto impar $N$, y el entero $x\in \mathbb{Z}^Q_N$. \\
	\textit{Pregunta:} & ¿Es $x$ un residuo cuadrático, $x \in \mathbb{Z}^{Q+}_N$? \\
\end{tabular}
\\

\hfil

Si $N$ fuera un número primo, el símbolo de Jacobi de $x$ coincidiría con el de Legendre, y la respuesta sería siempre $Verdadero$.

Si conocieramos la descomposición en primos de $N$, el algoritmo polinomial que resolvería el problema es calcular el símbolo de Legendre de $x$ respecto de cada factor primo de $N$, hasta encontrar alguno que valga $-1$, y responder al problema con $Falso$, o bien comprobar que todos los símbolos valen $1$ y responder $Verdadero$.

Con este algoritmo que utiliza la factorización de $N$, tenemos que el problema $QR$ puede \textbf{reducirse polinomialmente} (\ref{reducePoly:def}) al problema de factorización de $N$.

\begin{proposition}
	 $QR \leq_P FACTORIZACIÓN$
\end{proposition}

Si se desconoce la factorización de $N$, no se conoce a día de hoy ningún algoritmo eficiente para resolver el problema QR aparte del de intentar adivinar la respuesta. Por ejemplo, en el caso de $N=pq$, se tiene una probabilidad de acertar de $\frac{1}{2}$, por $\mid \mathbb{Z}_N^{Q+} \mid =$ $ \mid \mathbb{Z}_N^{Q-} \mid =$ $\frac{(p-1)(q-1)}{4}$ (\ref{numResCuadpq:prop}).

% TODO: referenciar sección 6 https://groups.csail.mit.edu/cis/pubs/shafi/1984-jcss.pdf , que da más detalles del QRAssumption

Del mismo modo que no se sabe si \textbf{P}=\textbf{NP} (aunque se cree que no), aquí se cree que $QR$ es tan difícil como el problema de factorización, pero no se conoce ninguna demostración aún. Bajo esta suposición, se construyen muchas aplicaciones criptográficas, entre ellas el cifrado de clave pública probabilístico de Goldwasser-Micali, el generador de números pseudo-aleatorios de Blum-Blum-Shub, o una de las pruebas de conocimiento cero más características, y que veremos en el capítulo siguiente.

%TODO : poner referencias a los ejemplos del párrafo anterior


%TODO: describir el problema de factorización y su mejor solución conocida