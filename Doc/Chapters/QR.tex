%************************************************
\chapter{Residuos Cuadráticos}\label{ch:qr} 
%************************************************

%TODO
TODO : Párrafo de introducción al capítulo y referencias
% Handbook of applied
% A course in computational


Teoría de símbolos de Lebesgue, ..., residuos cuadráticos, cálculo de raíz discreta?


%%%%%%


\begin{definition}
	Sea $a\in \mathbb{Z}^*_n$. Se dice que $a$ es un \textit{residuo cuadrático} módulo n, o un \textit{cuadrado} módulo n, si existe un $x \in \mathbb{Z}^*_n$ tal que $x^2 \equiv a \, mod \, n$.
	Si no existe dicho $x$, entonces $a$ se llama un \textit{no-residuo cuadrático} módulo n.
	
	El conjunto de todos los residuos cuadráticos módulo n de $\mathbb{Z}^*_n$ los denotaremos como $Q_n$ o bien como $\mathbb{Z}^{Q+}_n$.
	Al conjunto de los no-residuos cuadráticos lo denotamos como $\overline{Q_n}$.
\end{definition}

\begin{remark}
	Por definición $0 \notin \mathbb{Z}^*_n$, y por tanto $0 \notin Q_n$ y $0 \notin \overline{Q_n}$.
\end{remark}

\begin{proposition}
	Sea $p$ un primo impar. Se cumple que $|Q_p| = \frac{p-1}{2}$ y $|\overline{Q_p}| = \frac{p-1}{2}$, es decir, la mitad de los elementos de $\mathbb{Z}^*_p$ son residuos cuadráticos, y la otra mitad no-residuos cuadráticos.
\end{proposition}

\begin{proof}
	Sea $\alpha \in \mathbb{Z}^*_p$ un generador de $\mathbb{Z}^*_p$.
	Un elemento $a \in \mathbb{Z}^*_p$ es un residuo cuadrático módulo p sii $a \equiv \alpha^i \, mod \, p$ donde $i$ es un entero par. Como $p$ es primo, $\phi(p) = p-1 = |\mathbb{Z}^*_p|$, que es un entero par, y de ahí se sigue el enunciado.
\end{proof}


\section{Símbolos de Legendre-Jacobi-Kronecker}


\begin{definition}
	Dado un entero primo impar $p$, se define el \textit{símbolo de Lebesgue} 
\end{definition}
