%*****************************************
\chapter{Problemas Basados en Grafos}\label{ch:preliminaresGrafos}
%*****************************************

\section{Introducci\'on a la Teoría de Grafos}

\begin{definition}\index{Grafo}\index{Nodo}\index{Arista}
	Un \textit{grafo} (simple no dirigido) es un par $(V,E)$ formado por un conjunto
	finito $V \neq \emptyset$, a cuyos elementos denominaremos \textit{vértices} o
	\textit{nodos}, y $E$, un conjunto de pares (no ordenados y formados por distintos
	vértices) de elementos de $V$, a los que llamaremos \textit{aristas}.

	A los nodos $v_1$ y $v_2$ que forman una arista $e = (v_1, v_2)$ se les llama
	\textit{extremos} de $e$.
\end{definition}


\begin{definition}[Conceptos]\index{Ciclo}\index{Hamiltoniano}\index{Camino}
	\begin{itemize}
	\item A dos nodos $v_1$ y $v_2$ que forman parte de una misma arista se les llama \textit{adyacentes}.
  \item Se dice que una arista es \textit{incidente} con un vértice $v$ cuando $v$ es uno de sus extremos.
	\item El grado de un nodo $v \in V$ es el n\'umero de aristas incidentes en \'el.
  \item Un camino es una sucesi\'on de nodos $(v_1,v_2,\cdots,v_k)$ tales que
	para todo $i \in \{1,\cdots,k-1\}$ se tiene que $(v_i,v_{i+1}) \in E$.
	\item La longitud de un camino se define como el n\'umero de aristas que lo forman,
	en el caso del camino $(v_1,v_2,\cdots,v_k)$ la longitud es $k-1$.
	\item Un camino $(v_1,v_2,\cdots,v_k)$ se dir\'a un ciclo si $v_1 = v_k$ y no
	se repite ning\'un otro nodo del grafo, es decir, para todo $i,j \in
	\{1,\cdots,k-1\}$, si $i\not= j$ entonces $v_i \not= v_j$.
	\item Un ciclo $(v_1,v_2,\cdots,v_k)$ se dir\'a un ciclo hamiltoniano si contiene
	a todos los nodos.
	\item Un grafo diremos que es hamiltoniano si tiene un ciclo hamiltoniano.
\end{itemize}
\end{definition}

\begin{definition}
	Dos grafos $G_1 = (V_1, E_1)$ y $G_2 = (V_2, E_2)$ se dice que son \textit{isomorfos},
	lo cual denotaremos como $G_1 \simeq G_2$, si existe una aplicación biyectiva
	$\tau : V_1 \rightarrow V_2$ tal que $\forall u,v \in V_1$
	$(u,v) \in E_1 \Leftrightarrow (\tau(u), \tau(v)) \in E_2 $.
	Tal aplicación recibe el nombre de \textit{isomorfismo}.
\end{definition}

\begin{proposition}
Dados dos grafos isomorfos $G_1 = (V_1,E_1)$ y $G_2 = (V_2,E_2)$, entonces $G_1$
tiene un ciclo hamiltoniano si y s\'olo si $G_2$ tiene un ciclo hamiltoniano. Es
decir, la propiedad de ser un grafo hamiltoniano se conserva por isomorfismos.
\end{proposition}
\begin{proof}
Sea $\tau : V_1 \to V_2$ el isomorfismo y sea $(v_1,v_2,\cdots,v_k)$ el ciclo
hamiltoniano de $V_1$. Entonces $(\tau(v_1),\tau(v_2),\cdots,\tau(v_k))$ el el
ciclo hamiltoniano de $G_2$ ya que, por un lado es un ciclo puesto que
$(v_i,v_{i+1}) \in E_1$ implica que $(\tau(v_i),\tau(v_{i+1}))\in E_2$,
y por otro, al ser $\tau$ biyectiva, el ciclo contiene a todos los nodos
sin repetir ninguno (salvo los extremos).
\end{proof}

\begin{definition}\index{Coloraci\'on}
	Una \textit{coloración} de un grafo $G=(V,E)$ es una aplicación $\phi:V\rightarrow
	\{1,2,\dots , l\}$. El valor $\phi(v_i)$ es el \textit{color} correspondiente al
	nodo $v_i$.
\end{definition}

\begin{definition}
	Una \textit{coloración propia} de un grafo es una coloración que hace corresponder
	colores diferentes a vértices adyacentes, es decir
	$\forall (u,v)\in E$ se tiene que $\phi(u)\neq \phi(v)$.
\end{definition}


\begin{definition}
	Llamaremos \textit{número cromático} del grafo $G$, $\chi(G)$, al mínimo valor
	de $l$ que permite una coloración propia de $G$, es decir, al mínimo número de
	colores necesarios para colorear los vértices de forma que los extremos de cada
	arista tengan colores distintos.
\end{definition}

En las siguientes secciones de este cap\'itulo,
introducimos los enunciados de los problemas de grafos
utilizados en pruebas de conocimiento cero.

\section{El Problema del Isomorfismo de Grafos}

\hfil

\begin{tabular}{|ll}\label{problemaGI}\index{Problema Isomorfismo de Grafos}\index{GI}
	\textit{Nombre:} & Problema GI ({\em Graph Isomorphism}). \\
	\textit{Parámetros:} & Dos grafos $G_1 = (V_1, E_1)$ y $G_2 = (V_2, E_2)$,
	\\ & del mismo orden $\mid V_1 \mid = \mid V_2 \mid = n$. \\
	\textit{Pregunta:} & ¿Existe un isomorfismo $\tau : V_1 \rightarrow V_2$
	tal que \\ & una arista $(u,v)\in E_1$ si y solo si $(\tau (u),\tau (v))
	\in E_2$? \\
\end{tabular}
\\

\hfil

Una forma de tratar este problema por fuerza bruta consistir\'ia en calcular
todas las posibles premutaciones de $n$ elementos para ver si una de ellas
cumple las concidiones requeridas para el isomorfismo de grafos. Puesto que el
n\'umero de permutaciones de $n$ elementos es $n!$, esta aproximaci\'on ser\'ia
claramente exponencial e intratable para valores de $n$ m\'inimamente grandes.

Sin embargo, el problema puede tener casos en los cuales la soluci\'on sea
relativamente sencilla ya que los isomorfismos de grafos conservan muchas de las
propiedades del grafo. Por ejemplo, podemos calcular de forma muy r\'apida los
grados de todos los nodos y dado un isomorfismo $\tau$ tenemos que el grado de
$u$ es siempre el mismo que el grado de $\tau(u)$ para cualquier nodo $u$.

Esto nos puede restringir el problema a las permutaciones que se puedan construir
manteniendo los grados. Es algo parecido a lo que sucede con el algoritmo de la factorizaci\'on de enteros,
que en casos particulares puede tener una soluci\'on sencilla, pero debemos considerar
el caso peor para la utilizaci\'on de este problema como base de la seguridad de otros algoritmos.

El problema del isomorfismo de grafos se sabe desde hace tiempo que no es puramente
exponencial y que existen cotas subexponenciales, diferentes para algunos casos particulares,
pero ciertamente mejores que la exponencial.

Muy recientemente se han producido novedades en relaci\'on con este problema. Concretamente
en el a\~no 2015 el profesor László Babai present\'o un algoritmo quasi-polin\'omico para
el problema del isomorfismo de grafos. Esta publicaci\'on se puede consultar por ejemplo en \cite{babai}.
Sin embargo el 4 de enero de 2017, el propio autor se retractaba de su resultado al haber sido
detectado un error en la demostraci\'on. Poco despu\'es, el 9 de enero de 2017, anunciaba una versi\'on
corregida del problema que parece que se ha confirmado como correcta.

La complejidad que presenta este algoritmo es $e^{(\ln n)^{O(1)}}$. Las consecuencias que esto puede
tener en la utilizaci\'on del problema del isomorfismo de grafos en algoritmos de seguridad criptogr\'afica
es algo que escapa a nuestros conocimientos, aunque no parece que se haya producido ninguna reacci\'on
importante en relaci\'on con este resultado. En cualquier caso, conviene estar atentos a los avances
en este sentido por si debemos usar t\'ecnicas alternativas.

La historia de este descubrimiento se puede consultar en la revista Investigaci\'on y Ciencia,
edici\'on espa\~nola de la
revista Scientific American que est\'a disponible online en
\sloppy\url{http://www.investigacionyciencia.es/noticias/resuelto-el-problema-del-isomorfismo-de-grafos-otra-vez-14910}.

\hfil

\section{El Problema del Ciclo Hamiltoniano}

\hfil

\begin{tabular}{|ll}\label{problemaHC}\index{Problema Ciclo Hamiltoniano}\index{HC}
	\textit{Nombre:} & Problema HC ({\em Hamiltonial Cycle}). \\
	\textit{Parámetros:} & Un grafo $G=(V,E)$. \\
	\textit{Pregunta:} & ¿Existe un ciclo hamiltoniano en G? \\
\end{tabular}
\\

\hfil

De nuevo nos encontramos con un problema que podemos resolver considerando todas
las permutaciones de los nodos, ya que si existe una de ellas que nos proporcione
un ciclo hamiltoniano, habr\'iamos resuelto el problema. Pero este algoritmo es
claramente ineficiente puesto que requiere un an\'alisis de $n!$ permutaciones de los
$n$ nodos.

Una forma algo m\'as inteligente ser\'ia partir de un nodo cualquiera y crear una rama con cada uno de los nodos
adyacentes. Para cada uno de ellos, tratamos de expandir de nuevo con todos los nodos adyacentes siempre que
no repitamos el nodo. De esta forma podr\'iamos recorrer todas las opciones, pero volvemos a
encontrar un n\'umero de casos exponencial.

Este problema, sin embargo, est\'a en una situaci\'on muy distinta del problema anterior,
de hecho, el problema del ciclo hamiltoniano es un problema ${\bf NPC}$ por lo que si existiera un algoritmo
polin\'omico para resolverlo, tendr\'iamos que ${\bf P} = {\bf NP}$ y se habr\'ia resuelto uno de los problemas
del milenio.

La demostraci\'on de que este problema es ${\bf NPC}$ se puede ver en
\cite[Section 34.5.3]{algoBB}. El valor que nos sirve de certificado o testigo es precisamente el ciclo hamiltoniano, puesto que
comprobar que un ciclo concreto cumple las condiciones de ciclo hamiltoniano es trivial desde el punto de vista computacional.

\section{El Problema de la 3-coloración}

\hfil

\begin{tabular}{|ll}\label{problemaG3C}\index{Problema 3-coloración}\index{G3C}
	\textit{Nombre:} & Problema G3C. \\
	\textit{Parámetros:} &Un grafo $G=(V,E)$. \\
	\textit{Pregunta:} & ¿Existe una función
	$\phi : V \to \{1,2,3\}$ tal que \\ & $\forall (u,v)\in E$, se cumple $\phi(u)\neq \phi(v)$? \\
\end{tabular}
\\

\hfil

Determinar si un grafo puede tener una coloraci\'on dada por dos colores es
computacionalmente muy sencillo, puesto que no tenemos
mas que dar un color cualquiera al primer nodo, por ejemplo $1$, y todos los
que est\'en conectados con \'el le asignaremos el color $2$.
Ahora tomamos todos los nodos con el color $2$ y para cada uno de los nodos
adyacentes asignamos el color $1$. Si el grafo se puede colorear
con dos colores, este procedimiento nos ir\'a coloreando cada trozo del grafo
(en caso de que haya zonas desconectadas, cada vez que terminemos de colorear una zona,
tendremos que  partir aleatoriamente de nuevo de uno de los nodos de una nueva
zona para extender la coloraci\'on). Si el grafo no se puede colorear con dos colores
llegaremos necesariamente a un nodo coloreado con un color que est\'a unido con
otro del mismo color, en cuyo caso podemos responder
que el grafo necesita m\'as de dos colores para obtener una coloraci\'on.

Dado un grafo, este algoritmo nos responde de forma eficiente al problema de
si se puede colorear con dos colores, o bien devolviendo una coloraci\'on, o bien
diciendo que dicha coloraci\'on es imposible.

El problema es totalmente diferente en el caso de que dispongamos de tres colores.
Dado un nodo cualquiera con un color asignado, por ejemplo $1$, nos da dos opciones
para la coloraci\'on de los nodos adyacentes, o bien $2$ o bien $3$. Podemos tratar
de probar todas las opciones, con lo que obtendr\'iamos un gran n\'umero de posibilidades
cuando tratamos con un caso gen\'erico (por supuesto casos particulares pueden tener
soluciones sencillas).

De hecho es conocido que este problema tambi\'en pertenece a la clase {\bf NPC}. Una coloraci\'on
concreta usando $3$ colores servir\'ia de certificado de que la respuesta puede ser afirmativa puesto
que commprobar que todas las aristas tienen extremos con colores diferentes es sencillo
computacionalmente, sin embargo, encontrar la coloraci\'on requiere algoritmos exponenciales.

Una demostraci\'on de que el problema de la 3-coloraci\'on es {\bf NPC} se puede encontrar
en \cite[Problem 34-3]{algoBB}.

\hfil
