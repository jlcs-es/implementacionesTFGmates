%\part{Aplicaciones}


%************************************************
\chapter{Aplicaciones de las pruebas de conocimiento cero}\label{ch:aplicaciones} 
%************************************************


En este capítulo nos alejamos del estudio teórico de las pruebas de conocimiento cero para introducirnos en sus aplicaciones prácticas. La primera técnica que veremos consiste en transformar una prueba interactiva en no interactiva, aprovechando para generar un esquema de firma digital basado en pruebas de conocimiento cero. A continuación, veremos cómo se aplican las pruebas de conocimiento cero en un esquema de identificación de usuarios. Finalmente, como evolución de esos esquemas de identificación, estudiaremos cómo funcionan los certificados digitales de Identity Mixer diseñados por IBM y basados en pruebas de conocimiento cero. Se puede consultar la bibliografía de este capítulo en \citep{menezes1996handbook,stinson2005cryptography,idemixSpec}.

% menezes1996handbook
% Hacerlas no interactivas


\section{Pruebas de conocimiento cero no interactivas}

Hemos visto que la interacción habitual en una prueba de conocimiento cero consiste en enviar un \textit{testigo} $u$, un \textit{reto} $b$ y una \textit{respuesta} $w=\xi(u,b)$ en cada iteración, para que V verifique que el testigo corresponde al reto y respuesta enviados, $u=\vartheta(b,w)$.

La desventaja de estos protocolos en la práctica es sincronizar a P y V para comunicarse, por la necesidad de que el Verificador debe generar el reto él mismo, para estar seguro de que la prueba es correcta (por la propiedad de conocimiento cero). Para solucionarlo, querríamos generar pruebas de conocimiento cero de manera asíncrona, con un solo mensaje a ser verificado cuando V lo reciba. 

Una función de \textit{hash}\index{hash}, o resumen digital\index{Resumen digital}, recibe un mensaje arbitrario como entrada, y tras una serie de operaciones sobre el mismo, devuelve una cantidad fija de bits que es difícil de anticipar antes de ejecutar la función. Una propiedad de estas funciones es que con un mínimo cambio en el mensaje original, el resultado devuelto cambia drásticamente.

Una técnica para convertir las pruebas de conocimiento cero en no interactivas es la \textit{heurística de Fiat-Shamir}\index{Heurística Fiat-Shamir}\index{Fiat-Shamir}, la cual consiste en sustituir el reto $b$ por un resumen digital del testigo $u$, de modo que para un P computacionalmente limitado, éste no puede anticipar el valor del hash, y así V confía en que P no ha elegido el $u$ adecuado para falsear la prueba.

\hfil

Aprovechando la heurística de Fiat-Shamir, podemos convertir el protocolo en un esquema de firma digital\index{Firma digital} de un mensaje $m$, que equivale a una firma real sobre un documento de papel que sería el mensaje. Aplicando el resumen digital sobre el testigo y el mensaje a la vez, asociamos el mensaje a la prueba de conocimiento cero no interactiva:


\begin{center}
	\begin{tabular}{ll}\label{fiat-shamir-heur}
		P calcula :& el \textit{testigo} $u$, el \textit{reto} $h=$\textbf{hash(u|m)}, y la \textit{respuesta} $w = \xi(u, h)$.
		\\
		P $\rightarrow$ V :& \textit{\textbf{firma del mensaje}} $m$: \, $(h,w)$
		\\
		V verifica :& $h=hash(\,\vartheta(h,w)\,|\,m\,)$.
	\end{tabular}
\end{center}

\hfil


\begin{remark}
	\hfil
	
	En las pruebas de conocimiento cero perfectas estudiadas, los retos consistían en un bit, por lo que los posibles valores son $0$ ó $1$ y un ataque a la función de \textit{hash} es muy fácil. Por ello, se utilizan pruebas de conocimiento cero como el protocolo de Schnorr, donde el conjunto de los posibles valores del reto es suficientemente grande, usualmente unas centenas de bits.
\end{remark}








\section{Protocolos de identificación basados en pruebas de conocimiento cero}
% Fiat-Shamir, FFS, GQ, Schnorr


Las pruebas de conocimiento cero tienen una gran aplicación en el campo de la seguridad informática, en particular en la \textbf{autenticación} de usuarios, por ejemplo, al acceder al servidor de correo electrónico, o al aula virtual. Un sistema de identificación basado en pruebas de conocimiento cero consigue por un lado privacidad, al no revelar información del usuario, y por otro lado seguridad, al no \textit{degradarse} con el uso, es decir, resiste al criptoanálisis por muchos mensajes que se intercepten.

%Estos protocolos de identificación se basan en una prueba de conocimiento cero de un cierto problema $Q$, donde P (el usuario) tiene un \textit{secreto} que le permite demostrar una instancia $Verdadera$ al verificador V. Dicho valor secreto le relacionará con una identidad pública, de modo que al realizar la prueba, P se autentica como el dueño de dicha identidad.



\subsection{Protocolo de identificación de Fiat-Shamir}

El protocolo de identificación de Fiat-Shamir, basado en la prueba de conocimiento cero del problema QR, es el más característico junto con el de Schnorr (\autoref{honestVer:sec}).

Como nuestro usuario P no es computacionalmente todopoderoso, partiremos de que P conoce $s$, la raíz modular de un residuo cuadrático $v$, lo cual le permitiría realizar la prueba con éxito sin conocer la factorización de $N$. V autenticará al usuario P si éste es capaz de demostrar que $v$ es un residuo cuadrático, sin revelar $s$.


\rule{\textwidth}{1pt}
\begin{algorithm}[Protocolo de identificación Fiat-Shamir]
	\hfil
	
	\textit{Configuración de la identidad}:
	\begin{enumerate}
		\item La entidad de confianza selecciona y publica $N=pq$, con $p$ y $q$ primos y secretos.
		
		\item Cada usuario P genera un secreto $s \in \mathbb{Z_N^*}$, coprimo con $N$ (si no, se podría obtener la factorización de $N$ y perder la seguridad del protocolo). Calcula $v \equiv s^2 \, mod \, N$ y lo envía a la entidad de confianza como su clave pública.
		
	\end{enumerate}
	
	
	\textit{Protocolo}: Repetir $t$ rondas:
	\begin{enumerate}
		\item P escoge aleatoriamente $r \in_R \mathbb{Z_N^*}$, el \textit{compromiso}.
		\item $P \rightarrow V$:\quad $u \equiv r^2 \, mod \, N$, el \textit{testigo}.
		\item $V \rightarrow P$:\quad $b \in_R \{0,1\}$, el \textit{reto}.
		\item $P \rightarrow V$:\quad $w \equiv r\cdot s^b \, mod \, N$, la \textit{respuesta}.
		\item V verifica si \quad $ w^2 \equiv u\cdot v^b \, mod \, N$.
	\end{enumerate}
	
\end{algorithm}
\rule{\textwidth}{1pt}

\hfil


El protocolo de Fiat-Shamir es casi idéntico a la prueba interactiva \ref{QRinteractive:alg}, donde aquí $(v,N)$ hace el papel de instancia $Verdadera$ del problema QR, y como P, el usuario, es una máquina computacionalmente limitada, en vez de elegir aleatoriamente el residuo cuadrático $u$ y la raíz cuadrada $w$, parte de las raíces modulares $s$ y $r$ para poder calcular $u$, $v$ y $w$ residuos cuadráticos.


El intento de ataque que vimos en la demostración de robustez varía ligeramente para Fiat-Shamir, pues el objetivo aquí es intentar demostrar que $v$ es residuo cuadrático, sin conocer $s$ u otra raíz cuadrada, mientras que en la prueba interactiva del capítulo anterior, el objetivo era demostrar que un no-residuo cuadrático era un residuo cuadrático.

Un atacante debe adivinar el bit del reto que recibirá, de modo que si cree que $b=0$ calcula el testigo como en el protocolo, pero si cree que $b=1$, calcula en el paso 2 el testigo $u \equiv r^2\cdot v^{-1} \, mod \, N$, y en 4 la respuesta $w \equiv r \, mod \, N$. En ambos casos fallaría si no adivina $b$ correctamente, pues para corregir su error debería ser capaz de calcular, respectivamente, una raíz cuadrada de $v$, que permitiría pasar la prueba siempre, o una raíz cuadrada de $u \equiv r^2\cdot v^{-1} \, mod \, N$, computacionalmente inviable pues $p$ y $q$ los guarda como secretos la entidad de verificación.

Como en la prueba interactiva, un atacante tiene una probabilidad de $1/2$ de engañar a V en cada ronda, de modo que la robustez se mantiene con una probabilidad de ataque de $2^{-t}$. Si P pasa correctamente las $t$ rondas, V da por válida la prueba. Si falla aunque sea sólo una, rechaza la identificación.




%\subsection{Protocolo de identificación de Feige-Fiat-Shamir}
%
%% TODO: no estoy convencido de poner este. En el Fiat-Shamir el protocolo es prácticamente la prueba de conocimiento cero formal, aquí se intuye que se basa en la dificultad de QR, pero ya está.
%
%Una variación del protocolo de Fiat-Shamir para disminuir el número de mensajes intercambiados combinando varios testigos y retos a la vez.
%
%\rule{\textwidth}{1pt}
%\begin{algorithm}[Protocolo de identificación Feige-Fiat-Shamir]
%	\hfil
%	
%	\textit{Configuración de la identidad}:
%	\begin{enumerate}
%		\item \textit{Parámetros del sistema}: La entidad de confianza publica el módulo $N=pq$, con $p$ y $q$ $\equiv \, mod \, 4$, primos guardados secretos, de modo que $-1$ es un no-residuo cuadrático con símbolo de Jacobi 1. También define los enteros $k$ y $t$ que definen la seguridad.
%		
%		\item \textit{Selección del secreto}: Cada usuario P hace:
%		
%		\subitem (a) Selecciona aleatoriamente un vector $(s_1,s_2,\dots ,s_k)$, donde cada  $s_i \in \mathbb{Z_N^*}$ y $mcd(s_i, N)=1$. Además, elige también aleatoriamente $k$ bits $(b_1, b_2, \dots , b_k)$.
%		
%		\subitem (b) Calcula $v_i \equiv (-1)^{b_i}\cdot (s_i^2)^{-1} \, mod \, N$ para cada $i=1,\dots k$.
%		
%		\subitem (c) El usuario se identifica ante la entidad de confianza y envía su \textit{clave pública} $(v_1,\dots , v_k; N)$, y guarda su \textit{clave privada} $(s_1, \dots , s_k)$.
%		
%	\end{enumerate}
%	
%	
%	\textit{Protocolo}: Repetir $t$ rondas:
%	\begin{enumerate}
%		\item P escoge aleatoriamente $r \in_R \mathbb{Z_N^*}$ y un bit $b \in_R \{0,1\}$.
%		\item $P \rightarrow V$:\quad $u \equiv (-1)^b \cdot r^2 \, mod \, N$, el \textit{testigo}.
%		\item $V \rightarrow P$:\quad $(b_1,\dots ,b_k)$, con cada $b_i \in_R \{0,1\}$, el \textit{reto}.
%		\item $P \rightarrow V$:\quad $w \equiv r\cdot \prod_{j=1}^{k} s_j^{b_j} \, mod \, N$, la \textit{respuesta}.
%		\item V verifica si \quad $ u \equiv \pm \, w^2 \cdot \prod_{j=1}^{k} v_j^{b_i} \, mod \, N$.
%	\end{enumerate}
%	
%\end{algorithm}
%\rule{\textwidth}{1pt}
%
%La probabilidad de que un atacante pueda engañar a V es la de acertar el reto de $k$ bits en cada una de las $t$ rondas, es decir, $2^{-kt}$. Como vemos, aumentando el número de retos por ronda, podemos reducir el número de mensajes intercambiados para conseguir una robustez equivalente a Fiat-Shamir.



%
%\subsection{Protocolo de identificación de Schnorr}
%
%Vimos en la \autoref{honestVer:sec} que el protocolo de Schnorr, basado en el logaritmo discreto, era una prueba de conocimiento cero con verificador honesto. Aquí damos una ampliación del protocolo donde se indica cómo la entidad de confianza genera los certificados y cómo a partir de la prueba de Schnorr un usuario P se autentica ante V con su certificado y clave privada: 
%
%
%\rule{\textwidth}{1pt}
%\begin{algorithm}[Protocolo de identificación Schnorr]
%	\hfil
%	
%	\textit{Configuración de la identidad}:
%	\begin{enumerate}
%		\item \textit{Parámetros del sistema}: La entidad de confianza publica los parámetros $(p,q,\beta)$ donde $p$ y $q$ son primos tal que $q\mid (p-1)$, y $\beta$ es un elemento con orden multiplicativo $q$ (por ejemplo, para $\alpha$ un generador $mod\,p$, $\beta=\alpha^{(p-1)/q}\,\mod\,p$). Además se escoge un $t$, $2^t < q$, que definirá el nivel de seguridad.
%		
%				
%		\item \textit{Selección del secreto}: Cada usuario P:
%		
%		\subitem (a) Recibe una identidad única $I_P$.
%		
%		\subitem (b) Escoge una clave privada $a$, $0\leq a \leq q-1$, y calcula $v = \beta^{-a}\, mod\, p$.
%		
%		\subitem (c) La entidad certificadora vincula la identidad $I_P$ y el valor $v$ firmando, con cualquier método de firma $S_T(\cdot)$, el certificado $cert_P = (I_P, v, S_T(I_P|v))$.
%		
%	\end{enumerate}
%	
%	
%	\textit{Protocolo}:
%	\begin{enumerate}
%		\item P escoge aleatoriamente $r$, $0\leq r\leq q-1$, y un calcula $x=\beta^r\,mod\,p$.
%		\item $P \rightarrow V$:\quad $x$, el \textit{testigo}, y $cert_P$.
%		\item $V \rightarrow P$:\quad $e$ aleatorio, $1\leq e\leq 2^t<q$, el \textit{reto}, y verifica la firma del certificado.
%		\item $P \rightarrow V$:\quad $y=ae+r\, mod\, q$, la \textit{respuesta}.
%		\item V verifica si \quad $ x = \beta^y v^e \, mod \, p$.
%	\end{enumerate}
%	
%\end{algorithm}
%\rule{\textwidth}{1pt}




%
%
%
%
%





\section{Pruebas de conocimiento cero en Identity Mixer}

Identity Mixer\index{Identity Mixer}, o Idemix\index{Idemix} para acortar, es un protocolo desarrollado por IBM\footnote{Identity Mixer - \url{https://www.research.ibm.com/labs/zurich/idemix/}} para crear certificados digitales basados en atributos, donde se preserva la privacidad. Este protocolo no solo permite la autenticación sin revelar un valor secreto, también permite realizar pruebas sobre los atributos del certificado de un usuario, sin revelar los valores, como por ejemplo, ``\texttt{año\_actual - año\_nacimiento $\geq$ 18}''.

Los protocolos criptográficos en los que se basa Idemix \citep{idemixSpec} han sido desarrollados durante años por expertos en el área, donde destacan Jan Camenisch y Anna Lysyanskaya con el desarrollo de la firma CL \citep{camenisch2002signature} \citep{camenisch2001efficient} que se utiliza para firmar certificados sin tener que conocer los valores de todos los atributos, y que permite posteriormente al usuario demostrar que posee dicha firma, sin desvelarla. Esto permite utilizar múltiples veces un certificado de manera anónima sin que se relacionen los diferentes usos con la misma persona.





\subsection{Notación para ZKP}

Como hemos visto, cualquier problema de decisión posee una prueba de conocimiento cero, y de entre esas pruebas, las pruebas de conocimiento de un secreto son las que se utilizan en los certificados. Utilizaremos la siguiente notación de Camenisch y Stadler \citep{camenisch1997efficient} para denotar una prueba de conocimiento cero de conocimiento de un secreto.\index{ZKPoK}

\begin{center}
	$ZKPoK\{ (w) : \mathcal{L}(w,x) \}$
\end{center}
donde $w$ es un valor secreto, $x$ es información conocida por ambas partes, y $\mathcal{L}(w,x)$ es un predicado que representa una condición sobre $w$ y $x$. Se puede leer como ``conozco un secreto $w$ tal que el predicado $\mathcal{L}(w,x)$ se cumple para $w$ y $x$''. Para abreviar, en vez de ZKPoK (\textit{Zero-Knowledge Proof of Knowledge}), usaremos ZKP o PK.

Por ejemplo, $PK\{ (\alpha) : y=g^\alpha \}$ donde $y\in G = <g>$, un grupo de orden $q$ primo, denota el protocolo de identificación de Schnorr.


%\subsection{Combinar diferentes pruebas de conocimiento}
%
%Cuando leamos pruebas de conocimiento con varias condiciones, como
%\[	PK\{ (\alpha_1,\dots,\alpha_l,\beta_1,\dots,\beta_{l`}) : y=\prod_{i=1}^{l}g_i^{\alpha_i} \wedge z=\prod_{i=1}^{l`}h_i^{\beta_i} \} \]
%con $g_i,h_i,y,z\in G$, será equivalente a realizar de manera paralela e independiente cada protocolo:
%\[
%	PK\{ (\alpha_1,\dots,\alpha_l) : y=\prod_{i=1}^{l}g_i^{\alpha_i} \}\quad y\quad PK\{ (\beta_1,\dots,\beta_{l`}) : z=\prod_{i=1}^{l`}h_i^{\beta_i} \}
%\]
%donde en el primer mensaje P envía los testigos de cada uno de los protocolos, entonces, V envía un único reto, el mismo para cada protocolo, y finalmente, P responde al reto con la respuesta de cada prueba individual, y que V verificará.


\subsection{Probar el conocimiento de una representación}\label{reprKP}

Una generalización de Schnorr utilizada en Idemix es utilizar $l$ bases $g_1, \dots, g_l$ con $g_i \in G$, siendo $g$ un generador de $G$ con orden $ord(g)=q$ primo. Sea $y\in G$ tal que conocemos los valores $x_1,\dots,x_l$ que cumplen $y = \prod_{i=1}^{l} g_i^{x_i}$. Una prueba de conocimiento cero de que conocemos la representación de $y$ con respecto a las bases $g_1, \dots, g_l$ consiste en:


\rule{\textwidth}{1pt}
\begin{algorithm}
	\hfil
	
	P y V conocen $y$, $g_1, \dots, g_l$, $q$ y el parámetro del sistema $k$.
	
	\begin{enumerate}
		\item P escoge aleatoriamente $l$ enteros $r_i \in_R \mathbb{Z_q}$.
		\item $P \rightarrow V$:\quad $t=\prod_{i=1}^{l}g_i^{r_i}$.
		\item $V \rightarrow P$:\quad $c \in_R \{0,1\}^k$, un entero aleatorio de $k$ bits.
		\item $P \rightarrow V$:\quad $s_i = r_i - c\cdot x_i \, mod \, q$, para $i=1,\dots,l$.
		\item V verifica si \quad $t = y^c \prod_{i=1}^{l} g_i^{s_i}$.
	\end{enumerate}
	
\end{algorithm}
\rule{\textwidth}{1pt}

La probabilidad de que un P malicioso adivine los $k$ bits aleatorios del reto $c$ es de $2^{-k}$. Comprobamos que el protocolo funciona pues
\[
	y^c  \prod_{i=1}^{l} g_i^{s_i} =  \left( \prod_{i=1}^{l} g_i^{x_i} \right) ^c \prod_{i=1}^{l} g_i^{s_i} = \prod_{i=1}^{l} g_i^{c x_i + r_i -c x_i} = \prod_{i=1}^{l} g_i^{r_i} = t
\]




\subsection{Firma Camenisch-Lysyanskaya}

El esquema de firma de Camenisch-Lysyanskaya\index{CL}, o firma CL\index{Firma CL}, es la base de los certificados Idemix. Procedemos a describirlo.

Sean $p`$ y $q`$ dos primos, tal que $p=2p`+1$ y $q=2q`+1$ también lo son, llamados \textit{primos seguros}. Los parámetros del sistema, conocidos por el firmante y quien comprueba la firma son: el entero $n=pq$, $Z,S,R\in QR_n$ (grupo multiplicativo de los residuos cuadráticos módulo $n$). La clave secreta del firmante es $(p,q)$ y el mensaje lo denominaremos $m$. La firma CL sobre $m$ es la tupla $(A,e,v)$ tal que $A \equiv \left( \dfrac{Z}{S^v R^m} \right) ^{1/e} \, mod \, n$, donde $e,v$ son aleatorios, $e$ primo y $\frac{1}{e}\cdot e \equiv 1 \, mod \, \varphi(n)$.

Para verificar que una firma $(A,e,v)$ es correcta, basta comprobar si $Z \overset{?}{\equiv} A^e S^v R^m \, mod \, n$.

\hfil

La ventaja de esta firma es que se puede aplicar sobre múltiples mensajes a la vez, que en el caso de un certificado, serán los diferentes atributos a firmar. En este caso los parámetros del sistema se mantienen, $n=pq$, $Z,S\in QR_n$, pero debemos calcular un residuo cuadrático por cada mensaje a firmar, $R_0,\dots,R_l\in QR_n$. La clave secreta sigue siendo $(p,q)$, y la firma de los mensajes $m_0,m_1,\dots,m_l$ es $(A,e,v)$ tal que
\begin{center}
	$A \equiv \left( \dfrac{Z}{S^v \prod_{i=0}^{l} R_i^{m_i} } \right) ^{1/e} \, mod \, n$
\end{center}

\hfil

Para verificar la firma, comprobamos que $Z \overset{?}{\equiv} A^e S^v \prod_{i=0}^{l} R_i^{m_i} \, mod \, n$.

\subsection{Firma Camenisch-Lysyanskaya aleatorizada}\label{CLrandomized}

Dada una firma CL $(A,e,v)$, sobre los mensajes  $m_0,m_1,\dots,m_l$, podemos obtener una firma $(\acute{A},e,\hat{v})$ distinta de la anterior, excepto por $e$, que todavía sea válida. Esta nueva firma la puede generar cualquier usuario, pues no precisa del secreto $(p,q)$ de la firma original.

La nueva firma se obtiene a partir de un entero aleatorio $r$, calculando $\acute{A}= A\cdot S^{-r}\, mod\, n$ y $\hat{v}=v+er$.

La verificación de $(\acute{A},e,\hat{v})$ se realiza igual que en la firma original, ya que:

\begin{center}
	$
	\acute{A}^e S^{\hat{v}} \prod_{i=0}^{l} R_i^{m_i} \equiv A^e S^{-er} S^v S^{er} \prod_{i=0}^{l} R_i^{m_i} \equiv A^e S^v \prod_{i=0}^{l} R_i^{m_i} \equiv Z \, mod \, n
	$
\end{center}


\subsection{Firma de credenciales Idemix}\label{idemix:cert}

Una credencial\index{Credencial} Idemix consiste en una lista de atributos con valores, la cual estará firmada por un Emisor de confianza, como puede ser un gobierno al firmar el DNI electrónico. Sin embargo, la firma CL del certificado Idemix la calcularán interactivamente el Usuario y el Emisor, pues ninguno tendrá toda la información necesaria para generar la firma por sí mismo. En cada paso, una prueba de conocimiento cero asegurará que cada parte ha seguido el protocolo.

El Emisor conocerá la factorización de $n$, mientras que el Usuario guardará una clave secreta, $m_0$ y, opcionalmente, los atributos del certificado, $m_1,\dots,m_l$. Al final de la firma, el Usuario conocerá una representación, $(m_0,\dots,m_l)$ de $\frac{Z}{A^eS^v}$ con respecto a la base $(R_0,\dots,R_l)$, y como ya hemos visto, podrá demostrar con una prueba de conocimiento cero que conoce dicha representación sin revelar necesariamente ninguno. Además, el Emisor no conocerá la firma $(A,e,v)$ final, pues $v$ se calculará también de manera conjunta de modo que sólo el Usuario conocerá la firma generada.

\hfil

A continuación, presentamos una versión simplificada del protocolo, donde todos los atributos, menos la clave secreta, son conocidos por el Emisor. En otras versiones el Usuario puede enviar pruebas sobre alguna propiedad del atributo sin revelar el verdadero valor al Emisor, pero esto depende de la política del Emisor al firmar certificados. Es importante notar que el número de residuos cuadráticos generados por el emisor, $R_0,\dots,R_{l'}$ determina el máximo de atributos que puede llevar un certificado.

\rule{\textwidth}{1pt}
\begin{algorithm}[Emisión de certificado Idemix]
	\hfil
	
	\textit{Conocimiento común}: $n,Z,S,R_0,\dots,R_{l'},m_1,\dots,m_l$, donde $l'\geq l$
	
	\textit{Conocimiento del Emisor}: $p,q$ tal que $n=pq$.
	
	\textit{Conocimiento del Usuario}: $m_0$, su clave secreta.
	
	\hfil
	
	\textit{Protocolo}:
	\begin{enumerate}
		\item Ronda 1: Emisor
		\begin{enumerate}[label*=\arabic*.]
			\item El Emisor escoge un valor aleatorio $n_1$ y lo envía al Usuario.
		\end{enumerate}
		
		\item Ronda 2: Usuario
		\begin{enumerate}[label*=\arabic*.]
		
			\item El Usuario elige un valor $v'$ aleatorio (primer paso para calcular el $v$ de la firma).
			
			\item Calcula el valor $U = S^{v'} \cdot R_0^{m_0} \, mod \, n$.
			
			\item Calcula* la prueba $P_1 := ZKP\{(\nu',\mu_0) : U \equiv S^{\nu'}R_0^{\mu_0} \, mod \, n \}(n_1)$, no interactiva, dependiente de $n_1$.
			
			\item Envía $(U,P_1,n_2)$, donde $n_2$ es un valor aleatorio elegido por el Usuario.

		\end{enumerate}

		\item Ronda 3: Emisor firma CL
		\begin{enumerate}[label*=\arabic*.]
			\item Verifica* la prueba $P_1$ del Usuario.
			
			\item Elige un valor un $e$ primo y $v''$ aleatorios (segundo paso para calcular el $v$ de la firma).
			
			\item Calcula el valor $A = \left(  \frac{Z}{U\cdot S^{v''}\cdot \prod_{i=1}^{l} R_i^{m_i}} \right)^{1/e} \, mod \, n$.
			
			\item Calcula* $P_2 := ZKP\{(\delta) : A \equiv \left( \frac{Z}{U\cdot S^{v''}\cdot \prod_{i=1}^{l} R_i^{m_i}} \right)^{\delta} \, mod \, n \}(n_2)$, no interactiva, dependiente de $n_2$.
			
			\item Envía $(A,e,v'')$ y $P_2$ al Usuario.
		\end{enumerate}
		
		
		\item Ronda 4: Usuario
		\begin{enumerate}[label*=\arabic*.]
			\item Calcula $v=v'+v''$.
			\item Verifica $P_2$ y la firma CL $(A,e,v)$.
		\end{enumerate}

	\end{enumerate}
	
\end{algorithm}
\rule{\textwidth}{1pt}



\begin{flushleft}
	*Las pruebas y verificaciones correspondientes pueden ser consultados en el \autoref{ch:idemix}.
\end{flushleft}



\subsection{Revelación selectiva de atributos Idemix}

Tras la ejecución del siguiente protocolo, el Verificador conocerá algunos de los atributos del certificado, y una prueba de que el Emisor lo firmó, junto a unos atributos que no le mostramos, incluida la clave secreta $m_0$. El Verificador tampoco conocerá la firma CL original $(A,e,v)$, sólo $A'$ de una versión de la firma aleatorizada.

Partiendo de un certificado con la clave secreta $m_0$, los atributos $m_1,\dots,m_l$ y la firma CL $(A,e,v)$, vamos a revelar a un Verificador todos nuestros atributos excepto los dos primeros, $m_1$ y $m_2$, y por supuesto, tampoco la clave privada $m_0$.

Para esto, primero aleatorizaremos la firma CL como vimos en \ref{CLrandomized} y realizaremos una prueba no interactiva de que conocemos la representación $(m_0,m_1,m_2)$ (\ref{reprKP}) de un cierto valor en la base $R_0$, $R_1$ y $R_2$:

\rule{\textwidth}{1pt}
\begin{algorithm}[Revelación selectiva]
	\hfil
	
	\textit{Conocimiento común}: $n,Z,S,R_0,\dots,R_{l},m_3,\dots,m_l$.
	
	\textit{Conocimiento del Usuario}: $m_0$, su clave secreta, $m_1$ y $m_2$, sus atributos ocultos, $(A,e,v)$ su firma CL.
	
	\textit{Protocolo}:
	\begin{enumerate}
		\item V$\rightarrow$ P : valor aleatorio $n_1$.
		
		\item Usuario aleatoriza la firma CL:
		\begin{enumerate}[label*=\arabic*.]
			
			\item $r$ aleatorio, $(A':=AS^{-r}\,mod\, n,\, e,\, v':=v+er)$
			
		\end{enumerate}
		
		\item Usuario calcula la prueba de conocimiento no interactiva:
		
		$PK\{ (\hat{e},\hat{v}, m_0, m_1, m_2) :  \frac{Z}{\prod_{3}^{l}R_i^{m_i}} \equiv A'^{\hat{e}} S^{\hat{v}}\prod_{i=0}^{2} R_i^{m_i} \, mod \, n  \}(n_1)$
		
		\begin{enumerate}[label*=\arabic*.]
			\item Elige valores aleatorio $\tilde{e}$ y $\tilde{v}'$.
			\item Elige valores aleatorios $\tilde{m}_0$, $\tilde{m}_1$ y $\tilde{m}_2$.
			\item Calcula $\tilde{Z}:=(A')^{\tilde{e}} \left( \prod_{i=0}^3 R_i^{\tilde{m}_i} \right) S^{\tilde{v}'} \, mod \, n $.
			\item Genera el reto por Fiat-Shamir $c:=Hash(A'||\tilde{Z}||n_1)$.
			\item Calcula: $\hat{e}:=\tilde{e}+ce$;\; $\hat{v}':=\tilde{v}+cv'$;\; $\hat{m}_i:=\tilde{m}_i+cm_i$, para $i=0,1,2$.
			\item $P_1:=(c, \hat{e}, \hat{v}', \hat{m}_0, \hat{m}_1, \hat{m}_2)$.			
		\end{enumerate}
		
		\item U $\rightarrow$ V : $A'$ y $P_1$.
		
		\item Verificador comprueba:
		 \begin{enumerate}[label*=\arabic*.]
		 	\item Calcula $\hat{Z}:= \left( \frac{Z}{\prod_{i=3}^{l} R_i^{m_i}} \right)^{-c} (A')^{\hat{e}} \left( \prod_{i=0}^{3} R_i^{\hat{m}_i} \right)  S^{\hat{v}'} \, mod \, n$.
		 	\item Acepta si $c=Hash(A'||\hat{Z}||n_1)$.
		 \end{enumerate}
		
	\end{enumerate}
	

\end{algorithm}
\rule{\textwidth}{1pt}


\hfil

Un ejemplo cercano de uso de este protocolo podría ser las encuestas anónimas de asignaturas. Actualmente, si queremos asegurar al alumno que la encuesta es totalmente anónima, sin repercusiones, debemos dejarla \textit{abierta}, pudiendo cualquier persona rellenarla. Si queremos asegurar que sólo los alumnos matriculados puedan acceder a la encuesta, tendrían que iniciar sesión con su correo electrónico, y que confíen en que nadie accederá a sus comentarios.

Con un certificado Idemix, un atributo podría ser ``Alumno de la asignatura X'', otro sus datos de identificación. La Universidad firmó su certificado, que puede llevar en la tarjeta inteligente, por lo que disponemos de una firma CL. Al iniciar sesión en la encuesta, bastaría con mostrar que se es alumno de la asignatura, y generar la prueba no interactiva de que posee un certificado firmado por la Universidad. Sólo los verdaderos alumnos podrán acceder, y ningún registro puede relacionar una opinión con su autor.

Otras soluciones criptográficas tradicionales sólo se preocupaban de proteger la transmisión, que el mensaje no pudiera ser interceptado. Con las pruebas de conocimiento cero hemos conseguido proteger los mismos datos, ante un interlocutor en quien no confiamos.



% TODO: ZKP en multiparty computation: Idemix issuance

%\section{Pruebas de conocimiento cero en criptomonedas}
% O blockchain

% No dependes de los peers para asegurar tu privacidad, ni de una tercera entidad, ni de aplicar otras técnicas sobrela tecnología blockchain, las propias operaciones criptográficas aseguran la privacidad.

% Basecoin <-> Zerocoin : breaks link between new and original Basecoin

% Zerocoin := proof that you owned a Basecoin and made it unspendable
% Miners verify the proofs
% ZKP{ (m) : c==H(m) }  ;  ZKP{ (m) : c1==H(m) OR c2==H(m) OR c3==H(m) }


% Double spending: 
% Obtienes zerocoin: gastas un Basecoin por un zerocoin identificado por H(S,r), S y r secretos.
% ZKP{ (r) : H(S,r) == alguno de los zerocoins en la cadena } + verificar S no gastado antes (S se hace público, r se mantiene secreto).
% Al gastar el zerocoin, no se sabe a cual de los hashes era igual, como todos valen igual, se toma uno cualquiera

% N número de zerocoins en la cadena: tamaño de la prueba es logarítmico respecto N


% Zerocash: zerocoin sin Basecoin, y más eficiente
% No va y vuelve de basecoin a zerocoin, se puede trabajar todo el rato en zerocash
% All transactions are zerocoin
% Splitting and merging are supported. Put transaction value inside the envelope.
% Ledger merely records existence of transactions.









\clearpage

\section*{Conclusiones}
\addcontentsline{toc}{chapter}{Conclusiones}

Como cierre de este trabajo podemos destacar los conocimientos obtenidos durante el camino. Éste ha sido un trabajo de investigación que permite tratar un área de las matemáticas muy relacionada con la informática, desde que en el primer capítulo tratásemos la teoría de la computación hasta las aplicaciones de seguridad al final de la memoria. Sin embargo, los problemas matemáticos analizados y usados, como base de la criptografía y pruebas de conocimiento cero, no es una teoría que se estudie al mismo nivel en el Grado en Matemáticas que en otros estudios. El estudio formal que hemos realizado para poder entenderlas es fundamentalmente matemático, pero con aplicación a otras áreas.
 
 
Durante el seminario \textit{Grupos y criptograf\'ia: de Julio Cesar a las curvas el\'ipticas} de Adolfo Quirós, el pasado abril de 2017, se hizo el acertado comentario de que los problemas utilizados en criptografía no eran difíciles de entender. En efecto, hemos utilizado problemas de distintas áreas de las matemáticas, y los enunciados de los mismos no requerían un estudio extensivo sobre sus respectivas áreas para comprenderlos. Sin embargo, han demostrado que hasta el día de hoy, lo difícil es resolverlos. Por ejemplo, en el problema de la factorización vimos cómo las técnicas más avanzadas conseguían tiempos subexponenciales que sólo sirven en la práctica para factorizar números de tamaño relativamente pequeño.

También construimos unas herramientas fundamentales, los esquemas de compromiso, donde una de sus propiedades básicas, secreto o vinculación, dependían de la dificultad de resolver los problemas de teoría de números o grafos vistos.

Y también basándonos en esa dificultad, construimos algoritmos polinomiales que los pueden resuelven, a costa de suponer que una máquina P todopoderosa existe y es capaz de resolverlos, anulando en P la dificultad de los problemas. Modificando condiciones ajenas a P pasamos de pruebas perfectas, donde sabemos que de la prueba ninguna máquina podría obtener más información, a pruebas con Verificador Honesto, donde no conocemos ataques eficaces, o pruebas computacionales, donde la seguridad del conocimiento cero dependía de la dificultad del problema utilizado en el esquema de compromiso.

Sin embargo, si queremos utilizar dichos métodos en la práctica, nuestro P también estará limitado computacionalmente, y deberá conocer de antemano una solución al problema, sin tener que calcular la solución. Es lo que vimos en la variación de la prueba de conocimiento cero de los residuos cuadráticos en el algoritmo de identificación de Fiat-Shamir.

Finalmente, hemos visto que las aplicaciones de las pruebas de conocimiento cero no se quedan sólo en pequeñas interacciones para identificación, sino que permiten realizar tareas más complejas como computación distribuida, que se combinan en un protocolo como Idemix.

Inicialmente elegimos este área de trabajo debido a que para el TFG de Ing. Informática no había cabida para estudiar como se merecía la criptografía de Idemix, junto a su implementación en entornos IoT, tema central de dicho trabajo. Aprovechando que en el plan de estudios del doble grado ahora debíamos realizar dos Trabajos de Fin de Grado, separamos la teoría y la práctica de modo que no hubiera ningún solapamiento entre los trabajos presentados. Sin embargo, ambos se complementaron indirectamente, pues desarrollar uno ayudaba a comprender mejor el otro. En mi opinión personal, creo que fue una buena decisión que ahora refleja las ventajas del doble grado.



